\documentclass[a4paper,10pt,oneside]{article}
\usepackage[utf8]{inputenc}
\usepackage{setspace}
\usepackage{verbatim}
\usepackage{hyperref}
\usepackage{enumitem}
\usepackage{fontawesome}
\usepackage{url}
\hypersetup{colorlinks=true, urlcolor=blue}
\usepackage[left=2cm,right=2cm,top=1in,bottom=1in]{geometry}
\pagenumbering{arabic}
\usepackage{fancyhdr}
\pagestyle{fancy} 
\fancyhf{}
\fancyhead[L]{{\large Eric Rohr (he/him)}}
\fancyhead[R]{{\large \href{mailto:rohr@mpia.de}{rohr@mpia.de}} } 
\fancyfoot[R]{{\large \thepage} }
\fancypagestyle{firststyle}
{
\fancyhf{}
\renewcommand{\headrulewidth}{0pt}
\fancyfoot[R]{{\large \thepage}}
}
\begin{document}

\thispagestyle{firststyle}

\begin{center}
{\huge\textbf{Eric Rohr}} {\Large (he/him)} \\
\end{center}
\hrule 

\begin{center}
\begin{tabular*}{\textwidth}{l @{\extracolsep{\fill}}r}
{\large \hspace{-15pt} \faMapMarker\ Max Planck Institut f{\"u}r Astronomie} & {\large \faEnvelope\ \href{mailto:rohr@mpia.de}{rohr@mpia.de}} \\
{\large K{\"o}nigtushl 17} & {\large \faPhone\ +49 6221 528347} \\
{\large 69117, Heidelberg, Germany} & {\large \faGlobe\ \url{ecrohr.github.io}} \\
\end{tabular*}
\end{center}
\vspace{11pt}

\noindent\hspace{-.5cm}{\Large{\bf Reserach Interests and Keywords}}

\vspace{5.5pt}
\noindent A current list of all publications can be found on \href{https://ui.adsabs.harvard.edu/public-libraries/mi3QHw\_bSa6qgOlOj3c-sg}{ads} or \href{https://orcid.org/my-orcid?orcid=0000-0002-9183-5593}{ORCID}. 
\begin{itemize}[wide, labelwidth=!, labelindent=-11pt, parsep=0pt, itemsep=2pt]
    \item galaxy formation and evolution, cosmic baryon cycle, multiphase gas
    \item galaxy clusters, the intracluster medium, circumgalactic medium
    \item galaxy environments, ram pressure stripping, jellyfish galaxies
    \item star-formation, stellar feedvack, supermassive black hole formation, growth, and feedback
    \item numerical methods, cosmological hydrodynamic simulations, idealized simulations of galaxies
\end{itemize}

\vspace{5.5pt}

\begin{comment}
\noindent\hspace{-.5cm}{\Large{\bf Academic References and Advisors}}
\begin{itemize}[wide, labelwidth=!, labelindent=-11pt, parsep=0pt, itemsep=2pt]
    \item Dr. Annalisa Pillepich, MPIA, Ph.D. Advisor
    \item Dr. Dylan Nelson, ZAH, Thesis Advisory Committee
    \item Prof. Dr. Ralf Klessen, ZAH, Thesis Advisory Committee
    \item Prof. Dr. Hans-Walter Rix, MPIA, Director of Galaxies and Cosmology Department
    \item Dr. C\'eline P\'eroux, ESO, collaborator
    \item Prof. Robert Feldmann, UZH, former Supervisor
    \item Prof. Shane Davis, UVa, B.Sc Advisor
\end{itemize}

\vspace{5.5pt}
\end{comment}

\noindent\hspace{-.5cm}{\Large{\bf Education}} 

\vspace{5.5pt}

\noindent\begin{tabular*}{\textwidth}{p{4.5in} @{\extracolsep{\fill}} r}
    {\large Ph.D. Astronomy, 2024; Universit{\"a}t Heidelberg} & {\large October 2020-Present} \\
    \end{tabular*}
    \textit{Jellyfish Galaxies and the Multiphase Nature of Gas Around Galaxies}. Advisor: Dr. Annalisa Pilleich.
    \vspace{11pt}

\noindent\begin{tabular*}{\textwidth}{p{4.5in} @{\extracolsep{\fill}} r}
{\large B.Sc. Astronomy-Physics, 2020; {\it University of Virginia}} & {\large August 2016-May 2020} \\
\end{tabular*}
\textit{Why We Should Kerr About the Dark Secrets of Relativistic Accretion Disks in Athena++}. Advisor: Prof. Shane Davis.
\vspace{11pt}

\begin{comment}
\noindent\begin{tabular*}{\textwidth}{p{4.in} @{\extracolsep{\fill}} r}
    {\large Advanced Studies Diploma; {\it Atlee High School}} & {\large September 2012-June 2016} 
\end{tabular*}
\vspace{0pt}
\end{comment}

\noindent\hspace{-.5cm}{\Large{\bf Academic Appointments and Research Experience}} 

\vspace{5.5pt}

\noindent\begin{tabular*}{\textwidth}{p{4.5in} @{\extracolsep{\fill}} r}
    {\large Post-Doctoral Researcher*} & {\large December 2024-May 2025} \\
\end{tabular*}
Institut f\"ur Theoretische Astrophysik as part of the Zentrum f\"ur Astornomie in Heidelberg. Advisor: Prof. Dr. Ralf Klessen. Project: \textit{Zooming into the tails of jellyfish galaxies}. \\

\noindent\begin{tabular*}{\textwidth}{p{4.5in} @{\extracolsep{\fill}} r}
    {\large Ph.D. Student and IMPRS-HD Fellow} & {\large October 2020-November 2024} \\
\end{tabular*}
Max Planck Institut f{\"u}r Astronomie as a part of the International Max Planck Research School for Astronomy and Cosmic Physics at Universit{\"a}t Heidelberg; Advisor: Dr. Annalisa Pillepich. Thesis title: {\it Jellyfish Galaxies and the Multiphase Nature of Gas Around Galaxies}. \\

\noindent\begin{tabular*}{\textwidth}{p{4.5in} @{\extracolsep{\fill}} r}
    {\large Undergraduate Research Assistant} & {\large May 2019-July 2020} \\
\end{tabular*}
University of Virginia. Advisor: Prof. Shane Davis. Project: {\it Why We Should Kerr About the Dark Secretes of Relativistic Accretion Disks in Athena++}. \\

\begin{comment}
\noindent\begin{tabular*}{\textwidth}{p{4.5in} @{\extracolsep{\fill}} r}
    {\large Undergraduate Research Assistant} & {\large May 2019-May 2020} \\
\end{tabular*}
University of Virginia. Advisor: Prof. Shane Davis. Project: {\it Why We Should Kerr About the Dark Secretes of Relativistic Accretion Disks in Athena++}. \\
\end{comment}

\noindent\begin{tabular*}{\textwidth}{p{4.5in} @{\extracolsep{\fill}} r}
    {\large VSGC Undergraduate Research Scholar} & {\large August 2018-May 2019} \\
\end{tabular*}
University of Virginia as part of the Virginia Space Grant Consortium. Advisor: Prof. Mark Whittle. Project: {\it HST STIS Observations of the Central Radio/X-ray Source in the Compact Starburst Galaxy Henize 2-10}. \\

\noindent\begin{tabular*}{\textwidth}{p{4.5in} @{\extracolsep{\fill}} r}
    {\large ThinkSwiss Research Scholar} & {\large May 2018-August 2018} \\
\end{tabular*}
Universit{\"a}t Z{\"u}rich. Advisor: Prof. Robert Feldmann. Project: {\it Describing the Galaxy Size-Halo Size Relation at Cosmic Noon in FIREbox}. \\

\noindent\hspace{-.5cm}{\Large{\bf Publications}}

\vspace{5.5pt}

\noindent A current list of all publications can be found on \href{https://ui.adsabs.harvard.edu/public-libraries/mi3QHw\_bSa6qgOlOj3c-sg}{ads} or \href{https://orcid.org/my-orcid?orcid=0000-0002-9183-5593}{ORCID}. 

\vspace{5.5pt} 

\noindent {\bf As a first author:}
\begin{enumerate}[wide, labelwidth=!, labelindent=-11pt, parsep=0pt]
    \item[\href{https://datashare.mpcdf.mpg.de/s/fQsv8lkrESmya56}{4.}] \underline{Rohr, E.}, Pillepich, A., Nelson, D., et al. in review: ``The cooler past of the intracluster medium in TNG-Cluster''. MNRAS.
    \item[\href{https://ui.adsabs.harvard.edu/abs/2024A\%26A...686A..86R/abstract}{3.}] \underline{Rohr, E.}, Pillepich, A., Nelson D. et al. (2024): ``The hot circumgalactic media of massive cluster satellites in the TNG-Cluster simulation: existence and detectability''. A\&A, 686, A86.
    \item[\href{https://ui.adsabs.harvard.edu/abs/2023MNRAS.524.3502R/abstract}{2.}] \underline{Rohr, E.}, Pillepich, A., Nelson D. et al. (2023): ``Jellyfish galaxies with the IllustrisTNG simulations - when, where, and for how long does ram pressure stripping of cold gas occur?''. MNRAS, 524, 3502.
    \item[\href{https://ui.adsabs.harvard.edu/abs/2022MNRAS.510.3967R/abstract}{1.}] \underline{Rohr, E.}, Feldmann, R., Bullock, J. et al. (2022): ``The galaxy-halo size relation of low-mass galaxies in FIRE''. MNRAS, 510, 3967.
\end{enumerate}

\noindent {\bf As a contributing author:}
\begin{enumerate}[wide, labelwidth=!, labelindent=-11pt, parsep=0pt]
    \item[\href{https://www.aanda.org/articles/aa/abs/2024/10/aa48612-23/aa48612-23.html}{5.}] Ayromlou, M., Nelson, D., Pillepich A. et al. incl. \underline{Rohr, E.} (2024): ``An Atlas of Gas Motions in the TNG-Cluster Simulation: from Cluster Cores to the Outskirts''. A\&A, 690, A20.
    \item[\href{https://ui.adsabs.harvard.edu/abs/2024A\%26A...687A.129L/abstract}{4.}] Lehle, K., Nelson D., Pillepich A. et al. incl. \underline{Rohr, E.} (2024): ``The heart of galaxy clusters: demographics and physical properties of cool-core and non-cool-core halos in the TNG-Cluster simulation''. A\&A, 687, A129.
    \item[\href{https://ui.adsabs.harvard.edu/abs/2024A&A...686A.157N/abstract}{3.}] Nelson, D., Pillpeich, A., Ayromlou M. et al. incl. \underline{Rohr, E.} (2024). ``Introducing the TNG-Cluster Simulation: overview and physical properties of the gaseous intracluster medium''. A\&A, 686, A157.
    \item[\href{https://ui.adsabs.harvard.edu/abs/2024MNRAS.527.8257Z/abstract}{2.}] Zinger, E., Pillepich, A., Joshi, G. et al. incl. \underline{Rohr, E.} (2024): ``Jellyfish galaxies with the IllustrisTNG simulations - citizen-science results towards large distances, low-mass hosts, and high redshifts''. MNRAS, 527, 8257.
    \item[\href{https://ui.adsabs.harvard.edu/abs/2023MNRAS.525.3551G/abstract}{1.}] G{\"o}ller, J., Joshi, G., \underline{Rohr, E.} et al. (2023): ``Jellyfish galaxies with the IllustrisTNG simulations - No enhanced population-wide star formation according to TNG50''. MNRAS, 525, 3551.
\end{enumerate}

\noindent\hspace{-.5cm}{\Large{\bf Conferences, Talks, and Seminars}}

\vspace{5.5pt}

\noindent Given 30 talks among invited, contributed, and seminars and over the past five years. Selected highlights:

\begin{itemize}[wide, labelwidth=!, labelindent=-11pt, parsep=0pt]
    \item {\bf Invited Talk*}  {\it Resolving the Circumgalactic Medium and Its Impact on Galaxy Evolution} conference. Santa Cruz, Chile. November 2024.
    \item {\bf Invited Seminar*}  University of Zurich. Zurich, Switzerland. November 2024.
    \item {\bf Invited Talk*}  {\it Decade of Discovery: Celebrating 10 Years of the Illustris Project} workshop. Castello di Gargonza, Italy. November 2024.
    \item {\bf Invited Talk} Lunch Talk Series at the European Southern Observatory: ``Satellite Signatures in the Muliphase Halo Gas with the IllustrisTNG and TNG-Cluster Simulations''. Garching bei M{\"u}nchen, Germany. July 2024.
    \item {\bf Contributed Talk} {\it EAS Annual Meeting 2024}: ``Observable signatures from massive satellites in TNG-Cluster''. Remote in Padova, Italy. July 2024.
    %\item {\bf ePoster} at {\it EAS Annual Meeting 2024}: ``The hard lives of satellites galaxies in the TNG and TNG-Cluster simulations''. Remote in Padova, Italy. July 2024.
    \item {\bf Invited Talk} Galaxies Group Meeting at the Max Planck Institute for Astrophysics: ``Going out with a bang: the deaths of jellyfish galaxies in TNG50''. Garching bei M{\"u}nchen, Germany. April 2024.
    \item {\bf Invited Talk} Cosmology Seminar at the Max Planck Institute for Astrophysics: ``The case for the CGM around massive satellites in TNG-Cluster''. Garching bei M{\"u}nchen, Germany. April 2024.
    \item {\bf Invited Talk} Cosmic Baryon Cycle from Space workshop: ``The satellite-CGM connection in TNG''. Bern, Switzerland, February 2024.
    %\item {\bf Poster} {\it Building Galaxies from Scratch} conference: ``Comparing star formation and stellar feedback models in jellyfish galaxy bodies and tails''. Vienna, Austria. Februrary 2024.
    %\item {\bf Talk} at Galaxy Coffee at the Max Planck Institute for Astronomy: ``The case for the existence and detectability of the satellite circumgalactic media in TNG-Cluster''. Heidelberg, Germany. January 2024.
    \item {\bf Invited talk} Galaxy Coffee at the Institute of Astrophysics of the Canary Islands: ``Introducing the TNG-Cluster Simulation: the case for the circumgalactic medium around massive satellites''. La Laguna, Spain. Novevember 2023.
    \item {\bf Invited Talk} Galaxy Cluster Seminar at the Center for Astrophysics $\vert$ Harvard \& Smithsonian: ``Introducing the TNG-Cluster Simulation: the case for the circumgalactic medium around massive satellites''. Remote in Cambridge, Massachussetts, USA. Novevember 2023. 
    \item {\bf Contributed talk} {\it Journey through Galactic Environments} conference: ``Jellyfish galaxies as sources of cold gas in the CGM in the IllustrisTNG Simulations''. Porto Ercole, Italy. Septempber 2023.
    %\item {\bf Talk} at Galaxy Coffee at the Max Planck Institute for Astronomy: ``Understanding the CGM of massive satellite galaxies in the TNG-Cluster simulation''. Heidelberg, Germany. September 2023.
    %\item {\bf Talk} at Galaxy Coffee at the Maxk Planck Institute for Astronomy: ``Jellyfish galaxies with IllustrisTNG: when, where, and for how long does RPS of cold gas occur?''. Heidelberg, Germany. April 2023.
    %\item {\bf Poster} Saas Fee Winter School {\it Circum-Galactic Medium Across Cosmic Time}: ``The 5 W's of Ram Pressure Stripping in TNG Jellyfish''. Les Diableretes, Switzerland. March 2023.
    %\item {\bf Talk} at Galaxy Coffee at the Max Planck Institute for Astronomy: ``First steps towards jellyfish galaxies as probes of the cosmic gas''. Heidelberg, Germany. September 2022.
    \item {\bf Contributed talk} at {\it What Mattter(s) Around Galaxies} conference: ``Jellyfish galaxies with the IllustrisTNG simulations: when, where, and for how long does ram pressure occur, and implications for the cold CGM gas''. Champuloc, Italy. September 2022. \href{https://drive.google.com/file/u/0/d/1x4FNVCmUWTFwznOAs9RYhLwkKl-kmhZT/view?usp=drive_web}{Link to slides.}
    \item {\bf Contributed talk} {\it Epoch of Galaxy Quenching} conference: ``Jellyfish Galaxies with the IllustrisTNG simulations: when, where, and for how long does cold gas mass loss occur?''. Cambridge, United Kingdom. September 2022. \href{https://sites.google.com/cam.ac.uk/quenching/programme#h.ohkoyw4ilbje}{Link to talk.}
    %\item {\bf E-Poster} {\it Galaxy Clusters 2022} virtual conference: ``When, where, and how long do IllustrisTNG jellyfish galaxies take to lose their gas?''. Virtually held in Baltimore, Maryland, USA. April 2022.
    %\item {\bf Online Participant} KITP Program {\it Fundamentals of Gaseous Halos}. Held virtually in Santa Barbara, California, USA. January-March 2021. 
    %\item {\bf Invited Talk} Cosmic Dawn Center: ``Describing the Galaxy-Halo Size Relation at Cosmic Noon in FIREbox''. Copenhagen, Denmark. Februrary 2020. 
    %\item {\bf Invited Talk} University of Ghent: ``Describing the Galaxy-Halo Size Relation at Cosmic Noon in FIREbox''. Ghent, Belgium. Februrary 2020. 
    \item {\bf Contributed talk} {\it AAS 235 Winter Meeting 2020}: ``Describing the Galaxy-Halo Size Relation at Cosmic Noon in FIREbox''. Honolulu, Hawaii. January 2020. \href{https://ui.adsabs.harvard.edu/abs/2020AAS...23526001R/abstract}{Link to abstract.}
    %\item {\bf Poster} {\it AAS 233 Winter Meeting 2019}: ``HST STIS Observations of the Central Radio/X-ray Source in the Compact Starburst Galaxy Henize 2-10''. Seattle, Washington, USA. January 2019. \href{https://ui.adsabs.harvard.edu/abs/2019AAS...23335119R/abstract}{Link to abstract.}
    %\item {\bf Poster} {\it IAU Symposium 344 at General Assembly XXX}: ``HST STIS Observations of the Central Radio/X-ray Source in the Compact Starburst Galaxy Henize 2-10''. Vienna, Austria. August 2018. \href{https://doi.org/10.1017/S1743921318006282}{Link to conference proceedings.}
\end{itemize}


\noindent\hspace{-.5cm}{\Large{\bf Honors and Awards}}

\begin{itemize}[wide, labelwidth=!, labelindent=-11pt, parsep=0pt]
    \item {\it D. Nelson Limber Prize} from the Department of Astronomy at University of Virginia in May 2020. {\bf \$500.}
    \item {\it Alexander Vyssotsky Prize} from the Department of Astronomy at University of Virginia in May 2019. {\bf \$1,000.}
    \item {\it Undergraduate Research Scholarship} from the Virginia Space Grant Consortium, a division of NASA, to be taked at University of Virginia from August 2018-May 2019. {\bf \$4,000.} 
    \item {\it ThinkSwiss Research Scholarship} from the Office of Science, Technology, and Higher Eduacation at the Embassy of Switzerland, to be taken at University of Zurich from May-August 2018. {\bf 4,800 CHF}.
\end{itemize}

\noindent\hspace{-.5cm}{\Large{\bf Teaching and Mentoring}}

\begin{itemize}[wide, labelwidth=!, labelindent=-11pt, parsep=0pt]
    \item {\bf Co-Supervisor} of Fulbright Fellow Shalini Kurinchi-Vendhan at Max Planck Institute of Astronomy with Annalisa Pillepich, November 2023-Present
    \item {\bf Tutor} for the Fortgeschrittenenpraktikum Wellenfrontanalyzse (Advanced Lab on Wavefront Analysis; FP36) at University of Heidleberg. Winter Semester 2022-23.
    \item {\bf Assistant Tutor} at the Saas Fee Winter School {\it Circum-Galactic Medium Across Cosmic Time}. March 2023. 
    \item {\bf Tutor} for Cosmology (MVAstro4) at the University of Heidelberg. Summer Semester 2021, 2022.
    \item {\bf Teaching Assistant} for Observational Astronomy (ASTR3130) at the Universty of Virginia. Spring 2020.
    \item {\bf Tutor} for Advanced Placement (AP) Phsyics as part of the Global Teaching Project remotely teaching high school students in Mississippi. Fall 2019-Spring 2020.
    \item {\bf Teaching Assistant} for the undergraduate telescope observing lab at the University of Virginia. Fall 2017-Spring 2020. 
    \item {\bf Co-Instructor} for The Philosophical Implications of Physics (INST1550) at the University of Virginia. Spring 2019.
    \item {\bf Lab Assistant} for Elementary Physics Lab I and II (PHYS2630 and PHYS2640). Fall 2018-Spring 2019. 
\end{itemize}

\noindent\hspace{-.5cm}{\Large{\bf Service \& Outreach}}

\begin{itemize}[wide, labelwidth=!, labelindent=-11pt, parsep=0pt]
    \item Referee for MNRAS, A\&A, and AAS Journals. 2022-Present
    \item Student Representative for the 16th generation of IMPRS-HD students. Fall 2020-Fall 2024. 
    \item Member of the Internatational Max Planck Research School Board in Heidelberg. Spring 2022-Fall 2024.
    \item Published blog ``How Do the Tails of Jellyfish Galaxies Form'' on the Heidelberg \href{https://structures.uni-heidelberg.de/blog/posts/2024_04_jf/index.php}{STRUCTURES Blog}. April 2024.
    \item Organizer of Merendella (Happy Hour) at the Max Planck Institute for Astronomy. Summer 2021-2023.
    \item Published pub ``Quantifying how a jellyfish galaxy loses its cold gas'' on the \href{https://galacticatmospheres.pubpub.org/pub/8t27n1yz/release/1}{Galactic Atmospheres} forum. October 2023.
    \item Volunteer at Explore Science public day in Mannheim, Germany (in German). June 2022.
    \item Student Representative on the Graduate-Undergraduate Committee at the Department of Astronomy at the Universty of Virginia. Fall 2019-Spring 2020.
    \item Volunteer at the Leander McCormick Observatory Public Nights at the University of Virginia. Fall 2017-Spring 2020.
\end{itemize}

\noindent\hspace{-.5cm}{\Large{\bf Languages}}

\vspace{5.5pt}

\noindent \hspace{-19pt} \faKeyboardO\ {\bf Computer}: \texttt{Python} (expert), \texttt{C} (advanced), \texttt{C++} (advanced), Fortran (proficient), \texttt{html} (proficient)

\vspace{5.5pt}

\noindent \hspace{-17pt} \faLanguage\ {\bf Natural}: English (native), German (fluent, C1) \\

\begin{comment}
\vspace{11pt} 
{\it Last updated \today.}
\end{comment}

\end{document}
